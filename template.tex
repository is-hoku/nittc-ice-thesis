% A4, 二段組,フォントサイズ 9bp, 段間 2.36char
\documentclass[fontsize=9bp,twocolumn,column_gap=2.36zw,a4paper,report]{jlreq} 

\usepackage{nittc-ice}

\begin{document}
% 1 ページ目のみヘッダを挿入
\thispagestyle{header}
% タイトル(サブタイトル)とあらまし,キーワードは二段組にしない
\twocolumn[
	{\title 情報工学科論文\LaTeX テンプレート (タイトル)\par}
	{\subtitle{サブタイトル}\par}
	{\author{〇〇研究室}{豊田 高専}\par}
	{\abstract{これは豊田高専情報工学科の卒業論文用\LaTeX テンプレートです.提供されるdocxファイルとほぼ同様の見た目になるようにフォントや文字間隔や余白などを調整しました.Lua\LaTeX 処理系の使用を想定しています.}\par}
	{\keyword{\LaTeX, Word,豊田高専,卒業論文,テンプレート}}
]

% 和文文字間隔の設定
\ltjsetkanjiskip 0.165em plus 0.165em minus 0.165em
% 欧文和文文字間隔の設定
\ltjsetxkanjiskip 0.33em plus 0.33em minus 0.33em
% 欧文欧文文字感覚の設定
\addfontfeature{LetterSpace=5}

% 本文 (基本的にはタイトルとこれ以降の部分の編集をする)
\chapter{はじめに}

\section{\TeX とは}

\TeX はDonald Ervin Knuthが開発したオープンソースの組版システムです.
組版とは文字や図版,写真などをレイアウトの指定に従って配置する作業のことです.
\TeX と言った時,\TeX 処理系と\TeX 言語(マークアップ言語,プログラミング言語)の両方を指しています.
読み方は「テフ」もしくは「テック」です.

\section{\LaTeX とは}
\TeX の命令(プリミティブ)を組み合わせて作成したマクロを作成し,これを用いて文書を記述することが一般的です.
\LaTeX は,このマクロ体系の一つです.
読み方は「ラテフ」もしくは「ラテック」です.

\section{Lua\LaTeX とは}

スクリプト言語であるLuaが利用できる\LaTeX です.
通常,\TeX はDVIファイルを出力し,ユーザがこれをPDFファイルなどに変換するのですが,Lua\LaTeX ではPDFファイルを出力してくれます.
また,システムのOpenType/TrueTypeフォントが簡単に利用できたり,日本語の扱いが容易だったりという利点があります.
\newline

\chapter{環境構築}

まずLua\LaTeX をインストールします.
Lua\LaTeX と必要なパッケージを個別にインストールしても良いですが,TeX Liveをインストールすると一括で様々なパッケージやツールを利用できるようになるので楽ちんです.
また,「MS明朝」と「Times New Roman」をシステムのフォントとして配置します.
その後,このテンプレートをダウンロードしてコンパイルします.
\newline

\chapter{記述方法}

\texttt{\textbackslash chapter}や\texttt{\textbackslash section}でチャプターやセクションを書きます.
それぞれのチャプターやセクションに続けて本文を記述していきます.
このテンプレートではjlreqというクラスを使用しているため,書き方に関してはjlreqのドキュメント\cite{jlreq}を参照してください.

\section{ラベルと参照}

図や表,セクションなどを参照する場合は,参照される対象に\texttt{\textbackslash label}でラベルを記述し,参照する側で\texttt{\textbackslash ref}を記述します.\par
例えば\ref{fig:thesis}を参照する場合は,このように記述します.

\section{参考文献と引用}

文献の参照にBIB\TeX を使用します.
まず,\texttt{template.bib}に参照したい文献のデータベースを作ります.
このデータベースを参照できるようにするために,\texttt{pbibtex template}を実行します.
本文内で引用するときは\texttt{\textbackslash cite\{HIRANO2023301603\}}のように記述します.
最後に\TeX ファイルをコンパイルします.
参考文献は参照した順に自動で「文献」の下に挿入されます.
詳しい操作はこのサイト\cite{bibtex}を見てください.

\section{図の例}

このテンプレートでは,ページの区切りを明示的に記述しておらず,自動で改ページされるため,図表をページの特定の位置に挿入したい場合は記述位置を調整する必要があります.

\begin{figure}[h]
\centering{
\includegraphics[width=0.8\linewidth]{thesis.png}
\caption{卒業論文のイメージ}\label{fig:thesis}
}
\end{figure}

\section{ソースコードの例}

\begin{lstlisting}[caption=C言語の記述例,label=code:polyglot,language=c]
#include <stdio.h>
int main(int argc, char** argv)
{
#define B(x) x; printf("  B(" #x ")\n");
#define A(x) printf("  A(" #x ")\n"); x;
  B(printf("#include <stdio.h>\nint main(int argc, char** argv)\n{\n#define B(x) 
    x; printf(\"  B(\" #x \")\\n\");\n#define A(x) printf(\"  A(\" #x \")\\n\"); x;\n"))
  A(printf("}\n"))
}
\end{lstlisting}

\section{表の例}

\begin{table}[h]
	\caption{\TeX と\LaTeX}\label{tbl:texandlatex}
	\centering
	\begin{tabularx}{\linewidth}{|Y||c|c|}
		\hline
		& \TeX & \LaTeX \\
		\hline\hline
		開発者 & Donald Ervin Knuth & Leslie Lamport \\
		\hline
		発表年 & 1978年 & 1985年 \\
		\hline
		概要 & 言語と処理系 & マクロ体系 \\
		\hline
	\end{tabularx}
\end{table}

\section{数式の例}

\begin{equation}
	x=\frac{-b\pm\sqrt{b^2-4ac}}{2a}
\end{equation}

\chapter{免責事項}

このテンプレートを使用して生じたいかなる損害,不利益等について作者は一切の責任を負いません.
自己責任でお願いします.
\newline

\chapter{おわりに}

間違いや改善点があればコントリビュートしていただけると助かります.
\newline

\acknowledgement
本文書の執筆は,偉大な先達の知恵と努力によって成り立っています.
Knuth先生に敬意と感謝を申し上げます.
\newline

\bibliography{template}
\bibliographystyle{junsrt}

\end{document}
